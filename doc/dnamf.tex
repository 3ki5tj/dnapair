\documentclass{article}
\usepackage{amsmath}
\usepackage{amsthm}
\usepackage{amsfonts}
\usepackage{mathrsfs}
\usepackage{bm}
\usepackage[usenames,dvipsnames]{xcolor}
\usepackage{tikz}
\usepackage{hyperref}

\hypersetup{
    colorlinks,
    linkcolor={red!30!black},
    citecolor={blue!50!black},
    urlcolor={blue!80!black}
}


\begin{document}



\newcommand{\vct}[1]{\mathbf{#1}}
\newcommand{\vx}{\vct{x}}
\newcommand{\vy}{\vct{y}}
\newcommand{\Z}{\mathcal{Z}}
\newcommand{\E}{\mathcal{E}}
\newcommand{\Ham}{\mathcal{H}}
\newcommand{\W}{\mathcal{W}}
\newcommand{\A}{\mathcal{A}}
\newcommand{\LL}{\mathcal{L}}
\newcommand{\var}{\mathrm{var}}
\newcommand{\com}{\mathrm{com}}

\newcommand{\llbra}{[\![}
\newcommand{\llket}{]\!]}

% annotation macros
\newcommand{\repl}[2]{{\color{gray} [#1] }{\color{blue} #2}}
\newcommand{\add}[1]{{\color{blue} #1}}
\newcommand{\del}[1]{{\color{gray} [#1]}}
\newcommand{\note}[1]{{\color{OliveGreen}\small [\textbf{Comment.} #1]}}





%\title{DNA mean force}
\author{ \vspace{-10ex} }
\date{ \vspace{-10ex} }


%\maketitle

%\abstract{
%}

%\tableofcontents



\section{Mean-forces}


\subsection{General mean-force formula}


First, let us review a general mean-force formula.
%
For a quantity $X(\mathbf r)$
as a function of molecular configuration $\mathbf r$,
the distribution of $X$ is given by
$$
\rho(X)
=
\int \delta(X(\mathbf r) - X) \, w(\mathbf r) \, d\mathbf r,
$$
where the distribution
$$
w(\mathbf r)
=
\frac{ \exp[ - \beta \, U(\mathbf r) ] }
     { Z(\beta) }
.
$$
The standard mean-force formula in this case is
$$
\frac{ \partial \ln \rho(X) }
     { \partial X }
=
\left\langle
\nabla \cdot
  \left(
    \frac{ \mathbf V }
         { \mathbf V \cdot \nabla X }
  \right)
  +
  \frac{ \beta \mathbf V \cdot \mathbf F }
       { \mathbf V \cdot \nabla X }
\right\rangle_X
,
$$
where $\langle \dots \rangle_X$
means the average under a fixed value $X(\mathbf r) = X$,
and $\mathbf V$ is an arbitrary field of $\mathbf r$.

This formula can be generalized to a two-dimension distribution,
for
$$
\rho(X_1, X_2)
=
\int
\delta(X_1(\mathbf r) - X_1) \,
\delta(X_2(\mathbf r) - X_2) \,
w(\mathbf r) \, d\mathbf r,
$$
we have
$$
\frac{ \partial \ln \rho(X_1, X_2) }
     { \partial X_i }
=
\left\langle
\nabla \cdot
  \left(
    \frac{ \mathbf V_i }
         { \mathbf V_i \cdot \nabla X_i }
  \right)
  +
  \frac{ \beta \mathbf V_i \cdot \mathbf F }
       { \mathbf V_i \cdot \nabla X_i }
\right\rangle_{X_1, X_2}
,
$$
for $i = 1, 2$, as long as we can make sure
\begin{equation}
\mathbf V_1 \cdot \nabla X_2
=
\mathbf V_2 \cdot \nabla X_1
= 0
.
\label{eq:ortho}
\end{equation}



Below, we will use this formula to derive
the mean-forces along $R$ and $\Theta$
of a pair of DNAs.


\subsection{Radial mean force}


For the radial force between the $x$-separated two helices,
we define their separation
between the centers of mass
$$
R(\mathbf r) = X_+(\mathbf r) - X_-(\mathbf r)
,
$$
where $+$ and $-$ are the labels of the two helices,
and
\begin{equation}
X_s(\mathbf r)
=
\frac{ m_{s1} X_{s1} + \cdots + m_{sn} X_{sn} }
     { M_s }
,
\label{eq:comx}
\end{equation}
where, $s = +$ or $-$,
the total mass $M_s = m_{s1} + \cdots +m_{sn}$,
and $n$ is the number of atoms on each helix.

The gradient $\nabla X$ is
$$
\nabla R_{si}
=
\left( \frac{ \partial R } { \partial \mathbf r_{si} } \right)
=
\left(
  s \, \frac{ m_{si} } { M_s } , 0, 0
\right)
,
$$
for the $i$th atom of helix $s$.
%
A convenient vector field is
\begin{equation}
  \left( \mathbf V_R \right)_{si}
  =
  (s \, M_s, 0, 0)
  ,
\end{equation}
such that
$\mathbf V_R \cdot \nabla R = \sum_{s, i} m_{si} \, s^2 = M_+ + M_- = 2 \, M_s$,
and
$$
  \left( \frac{ \mathbf V_R } { \mathbf V_R \cdot \nabla R } \right)_{si}
  =
  \left( \frac s 2, 0, 0 \right)
  ,
  \qquad
  \nabla \cdot
  \left( \frac{ \mathbf V_R } { \mathbf V_R \cdot \nabla R } \right)_{si}
  = 0.
$$
Besides,
$$
  \frac{ \mathbf V_R } { \mathbf V_R \cdot \nabla R } \cdot \mathbf F
  =
  \sum_{s, i} F_{six} \frac{ s } { 2 }
  .
$$
So we get
$$
\frac{ \partial \ln \rho(R, \Theta) }
     { \partial R }
=
\frac \beta 2
\left\langle
  \sum_{i = 1}^n F_{+ix}
  -
  \sum_{i = 1}^n F_{-ix}
\right\rangle_{R, \Theta}
.
$$


\subsection{Angular mean force}

Next, consider the angular mean force.
$$
\Theta(\mathbf r) = \Theta_+(\mathbf r) - \Theta_-(\mathbf r)
,
$$
where
$$
\Theta_s(\mathbf r)
=
\frac{ I_{s1} \theta_{s1} + \cdots + I_{sn} \theta_{sn} }
     { I_s }
,
$$
and
$$
\begin{aligned}
I_{si}
&=
m_{si}
\left[
  (x_{si} - X_s)^2
  +
  (y_{si} - Y_s)^2
\right],
\\
I_s
&= \sum_{i = 1}^n I_{si},
\\
\theta_{si}
&=
\tan^{-1}
\frac{ y_{si} - Y_s }
     { x_{si} - X_s }
,
\end{aligned}
$$
with the $y$ coordinate of the center of mass, $Y_s$,
defined similarly as by Eq. \eqref{eq:comx}.
%
Below, in deriving the mean force,
we shall, however, treat $X_s$ and $Y_s$
as fixed.

Now since
$$
\nabla \theta_{si}
=
\frac{
  \bigl(
   -(y_{si} - Y_s),
   x_{si} - X_s,
   0
  \bigr)
}
{ (x_{si} - X_s)^2 + (y_{si} - Y_s)^2 }
,
$$
the gradient is
$$
\left( \nabla \Theta \right)_{si}
=
\frac{ s \, m_{si} }
{ I_s }
  \bigl(
   -(y_{si} - Y_s),
   x_{si} - X_s,
   0
  \bigr)
,
$$
and we choose the conjugate field as
\begin{equation}
\left( \mathbf V_\Theta \right)_{si}
=
s \, I_s \,
  \bigl(
   -(y_{si} - Y_s),
   x_{si} - X_s,
   0
  \bigr)
.
\end{equation}
In this way,
the orthogonality Eq. \eqref{eq:ortho}
is satisfied.

So $\mathbf V_\Theta \cdot \nabla \Theta = I_+ + I_- = 2 \, I_s$,
and
$$
  \left( \frac{ \mathbf V_\Theta } { \mathbf V_\Theta \cdot \nabla \Theta } \right)_{si}
  =
  \frac s 2
  \bigl(
   -(y_{si} - Y_s),
   x_{si} - X_s,
   0
  \bigr)
  ,
$$
leading to
$$
  \nabla \cdot
  \left( \frac{ \mathbf V_\Theta } { \mathbf V_\Theta \cdot \nabla \Theta } \right)_{si}
  = 0.
$$
Besides,
$$
  \frac{ \mathbf V_\Theta }
  { \mathbf V_\Theta \cdot \nabla \Theta } \cdot \mathbf F
  =
  \sum_{s, i} \tau_{siz} \frac{ s } { 2 }
  ,
$$
where the torque $\tau_{siz}$ defined as
$$
\tau_{siz}
=
(x_{si} - X_s) F_{siy}
-
(y_{si} - Y_s) F_{six}
.
$$
So we get
$$
\frac{ \partial \ln \rho(R, \Theta) }
     { \partial \Theta }
=
\frac \beta 2
\left\langle
  \sum_{i = 1}^n \tau_{+iz}
  -
  \sum_{i = 1}^n \tau_{-iz}
\right\rangle_{R, \Theta}
.
$$



\section{Determining the two-dimensional PMF from the mean forces}


To determine the mean force $\phi_i$ on a few grid points $i$,
we will minimize the functional
$$
E
=
\frac 1 2
\sum_{(i, j)}
\frac{ 1
}
{
  \mathrm{Var} (f_i) + \mathrm{Var} (f_j)
}
  \left(
  \frac {\phi_i - \phi_j} { \Delta x_{ij} }
  +
  \frac{ f_i + f_j } { 2 }
  \right)^2
,
$$
where
$\Delta x_{ij}$ is the difference in $R$
in a distance-separated pair
or the difference in $\Theta$
in an angle-separated pair,
$f_i$ and $f_i$ means the mean force
along $R$ for a distance-separated pair
or the mean force along $\Theta$
for an angle separated pair,
and
$\mathrm{Var}(f_i)$
is the variance of $f_i$.

Thus, to minimize the function, we have
\begin{equation}
0
=
\frac{ \partial E } { \partial \phi_i }
=
\sum_{j \mathrm{\; n.n. \; to\; } i }
\frac{
  1
}
{
  \mathrm{Var} (f_i) + \mathrm{Var} (f_j)
}
\left(
  \frac {\phi_i - \phi_j} { \Delta x_{ij}^2 }
  +
  \frac{ f_i + f_j } { 2 \, \Delta x_{ij} }
\right)
,
\label{eq:phieqs}
\end{equation}
where the sum is carried over neighbors of $i$.
These equations are not independent, however.
%
So we need to supply an additional equation,
such as $\phi_i = 0$ or $\sum_i \phi_i = 0$.
%
By solving Eq. \eqref{eq:phieqs} and this additional equation,
we get the PMF $\var_i$.

%\bibliographystyle{plain}
%\bibliography{simul}

\end{document}
